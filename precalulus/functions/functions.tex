\documentclass{article}
\usepackage{hyperref}
\title{Functions}
\author{Hammad Majid}
\date{2/11/22}

\begin{document}

\maketitle

\section*{Introduction}

\textbf{Definition:} A Function is correspondence between input  numbers $x-values$ and output numbers $y-values$ than sends each input number $x-value$ to exactly one output number $y-value$
\linebreak
Sometimes a function is described with an equation

\textbf{Example:} $y = x^2 + 1$ which can also be written as $f(x) = x^2 + 1$
\linebreak
What is? f(2)?

solution:
\linebreak
put 2 in $f(x) = x^2 + 1$
\linebreak
so it becomes 
$$f(2) = 2^2 + 1$$
$$f(2) = 4 + 1$$
$$f(2) = 5$$

what is f(a+3)?
\linebreak
put $a + 3$ into $f$
$$f(a + 3) = (a + 3)^2 + 1$$
$$f(a + 3) = a^2 + 3^2 + 2.a.3 + 1$$
$$f(a + 3) = a^2 + 9 + 6a + 1$$
$$f(a + 3) - a^2 + 6a + 10$$
\end{document}

